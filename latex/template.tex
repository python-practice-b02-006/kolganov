\documentclass{beamer}
\usepackage[T2A]{fontenc}
\usepackage[utf8]{inputenc}
\usepackage[russian]{babel}
\usepackage{graphicx}
\usepackage{color}
\usepackage{minted}

\usetheme{Madrid}
\usefonttheme[onlymath]{serif}



\title[Пр-ия о пр-ии]{Презентация о презентации} 
\subtitle{Маленький такой подзаголовочек}
\author{Никита Колганов}
\institute[МФТИ]
{Московский физико-технический институт\\
\textit{nikita.kolganov@phystech.edu}}
\date[\today]{Уже \today, пора начинать ботать}

\begin{document}

\begin{frame}
	\titlepage
\end{frame}


\begin{frame}[fragile]{Как устроен этот файл}
DISCLAIMER: Если вам не интересно, как и почему всё работает, пропустите этот слайд. \medskip \\ 
Как и каждый латеховский проект, этот документ начинается с шапки

\begin{minted}{latex}
%класс документа - beamer, специальный класс для презентаций
\documentclass{beamer}        
%далее идут импорты необходимых пакетов
\usepackage[T2A]{fontenc}    %шрифт
\usepackage[utf8]{inputenc}    %кодировка
\usepackage[russian]{babel}    %обработка русского текста
\usepackage{graphicx}    %графика
\usepackage{minted}    %листинги кода, типа этого
\end{minted}
\end{frame}


\begin{frame}[fragile]{Информация о докладчике и докладе}

Далее идут команды, отображаемые на титульном слайде вашей презентации
\begin{minted}{latex}
\title[Короткое название]{Название презентации} 
\subtitle{Подзаголовок презентации}
\author[Сокращенное имя докладчика]{Имя докладчика}
\institute[Короткое название организации]
        {Полное название организации}
\date[Короткая дата]{Дата}    %можно поставить \today
\end{minted}
Здесь и везде -- выражения в квадратных скобках -- необязательные, их можно не заполнять (как и сами скобки).
\end{frame}

\begin{frame}[fragile]{Настройка темы презентации}
Следующая команда задаёт тему презентации. 
Они называются по городам. 
Я рекомендую Warsaw и Madrid (никогда там не был).
\begin{minted}{latex}
\usetheme{Madrid}
\end{minted}
В презентациях принято использовать шрифты ``без засечек'', они хорошо читаются издалека. 
Однако формулы, написанные шрифтом без засечек выглядят отвратительно. Исправить это недоразумение можно следующей командой.
\begin{minted}{latex}
\usefonttheme[onlymath]{serif}
\end{minted}
\end{frame}

\begin{frame}[fragile]{Тело документа}
Далее начинается ``тело документа'' --- то, что будет отображаться
\begin{minted}{latex}
\begin{document}

...

\end{document}
\end{minted}
Все, что написано вне этого окружения по умолчанию игнорируется. \medskip \\
Вообще, примерно так выглядят все окружения в \LaTeX
\begin{minted}{latex}
\begin{имя окружения}[необязательные аргументы]
                    {обязательные (если предусмотрены)} 
    Какая-то начинка
\end{имя окружения}
\end{minted}
С другими окружениями мы познакомимся на следующем слайде.
\end{frame}

\begin{frame}[fragile]{Слайд о слайдах}
Чтобы создать слайд, нужно (внутри окружения document) написать
\begin{verbatim}
\begin{frame}{Название слайда}
Содержание слайда
\end{frame}\end{verbatim}
\end{frame}

\begin{frame}[fragile]{Слайд о первом слайде}
Чтобы создать титульный слайд (с информацией о докладе и докладчике),
 нужно использовать команду titlepage внутри окружения слайда frame
\begin{verbatim}
\begin{frame}
\titlepage
\end{frame}\end{verbatim}
В этом случае указывать название слайда не нужно.
\end{frame}

\begin{frame}[fragile]{Колоночки}
Чтобы разбить слайд на колонки, нужно использовать окружение columns
\begin{minted}{latex}
\begin{columns}
\column{0.6\textwidth} Колонка, шириной 0.6 от ширины слайда. 
Этот текст является содержимым первой колонки.

\column{0.4\textwidth} Колонка, шириной 0.4 от ширины слайда
Этот текст является содержимым второй колонки.
\end{columns}
\end{minted}
Результат --- на следующем слайде.
\end{frame}

\begin{frame}[fragile]{Колоночки --- демонстрация}
Текст до колоночек
\begin{columns}
\column{0.4\textwidth} Колонка, шириной 0.4 от ширины слайда. 
Этот текст является содержимым первой колонки.

\column{0.5\textwidth} Колонка, шириной 0.5 от ширины слайда
Этот текст является содержимым второй колонки.
\end{columns}
Текст после колоночек
\end{frame}

\begin{frame}[fragile]{Картиночки}
Для добавления картинок используется окружение figure. Но оно создаёт только поле для картинки/её названия, для добавления самой картинки нужно использовать includegraphics
\begin{columns}
\column{0.7\textwidth} Код:
\begin{minted}{latex}
\begin{figure}
\centering %центрирование
\includegraphics[width=0.7\textwidth]
    {abacaba.png}
\caption{Подпись к картинке}
\end{figure}
\end{minted}
Мы задействовали необязательный аргумент includegraphics, настраивающий ширину картинки.
\column{0.3\textwidth} Результат:
\begin{figure}
\centering
\includegraphics[width=0.7\textwidth]{abacaba.png}
\caption{Подпись к картинке}
\end{figure}
\end{columns}
\end{frame}

\begin{frame}[fragile]{Листинги кода}
А теперь один из самых важных моментов для презентации вашего проекта --- листинги кода.
Чтобы beamer не портил форматирование кода, необходимо использовать специальное окружение --- verbatim. Оно воспроизводит содержащийся в нём текст ``буквально''
\begin{verbatim}\begin{verbatim}
Текст, написанный внутри этого окружения
      воспроизводится буквально, с сохранением
            пробелов   
   и  переносов
\begin{verbatim}\end{verbatim}
Однако, чтобы использовать это окружение внутри слайда, надо обязательно использовать необязательный аргумент fragile!
\begin{verbatim}
\begin{frame}[fragile]{<-- необязательный аргумент тут}
Содержание слайда
\end{frame}\end{verbatim}
\end{frame}

\begin{frame}[fragile]{Листинги кода -- upper intermediate}
Есть специальный пакет --- minted, который не только сохраняет форматирование кода, но и подсвечивает его, согласно синтаксису конкретного языка. \medskip \\
Так выглядит код:
\begin{verbatim}\begin{minted}{python}
def funktsiya():
    return "tak nazyvat' funktsii nel'zya"
\end{minted}\end{verbatim}
А так --- результат:
\begin{minted}{python}
def funktsiya():
    return "tak nazyvat' funktsii nel'zya"
\end{minted}
Вместо python можно указать другой язык, например C++ или latex.
\end{frame}

\begin{frame}[fragile]{minted --- не всё так просто}
Если вы пользуетесь overleaf.com, то латеховский код, использующий окружение minted скомпилируется без проблем.
Однако, чтобы всё заработало в miktex, необходимо лезть в настройки, а именно:
\begin{enumerate}
\item Нажать ``Правка'' --- ``Настройки''
\item Во вкладке ``Вёрстка'' в списке ``Инструменты вёрстки'' выбрать активный профиль вёрстки (скорее всего это pdfLaTeX+MakeIndex+BibTeX --- можете посмотреть справа от зелёной стрелочки --- кнопки, которой вы запускаете компиляцию вашего латеховского проекта)
\item Справа от списка нажать ``Правка''. В появившемся окне нажать плюсик, после чего в списке ``Параметры'' нужно вписать ``\verb|--tex-option=--shell-escape|''. Выходите из окошек кнопкой ``OK''
\item Готово, можно пользоваться minted
\end{enumerate}
\end{frame}

\begin{frame}[fragile]{Списки}
Кстати, на предыдущем слайде появился список --- для него тоже есть своё окружение enumerate. Очередной элемент списка начинается с команды item. \medskip
\begin{columns}
\column{0.45\textwidth} Код:
\begin{minted}{latex}
\begin{enumerate}
\item Текст первого пункта
\item Текст второго пункта
\item Надоело
\end{enumerate}
\end{minted}
\column{0.45\textwidth} Результат:
\begin{enumerate}
\item Текст первого пункта
\item Текст второго пункта
\item Надоело
\end{enumerate}
\end{columns} \medskip
Для ненумерованного списка вместо enumerate напишите itemize.
\end{frame}

\begin{frame}[fragile]{Формулки}
Вообще, \LaTeX придумал великий математик Дональд Кнут, чтобы писать в нём красивые формулки. Формулы делятся на внутристрочные и вынесенные. Первые выглядят так $S=\pi r^2$, а вторые --- так
$$
G(x-x') = \int \frac{d^4 p}{(2 \pi)^4} \frac{e^{i p_\mu (x - x')^\mu}}{p_\nu^2 - m^2 + i \epsilon}
$$
Внутристрочные  выделяются одиночными знаками доллара \mintinline{latex}{$S=\pi r^2$}, а вынесенные --- двойными
\begin{minted}{latex}
$$
G(x-x') = \int \frac{d^4 p}{(2 \pi)^4} 
\frac{e^{i p_\mu (x - x')^\mu}}{p_\nu^2 - m^2 + i \epsilon}
$$
\end{minted}
Научиться писать красивые формулы не является целью этой презентации. Однако общий смысл и правила можно почерпнуть из примеров выше. Для большего есть прекрасные руководства, в том числе на русском языке.
\end{frame}

\begin{frame}
\Huge \centering Спасибо за понимание:)
\end{frame}

\end{document}