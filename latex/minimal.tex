\documentclass{beamer}      %класс документа - beamer, специальный класс для презентаций

%далее идут импорты необходимых пакетов
\usepackage[T2A]{fontenc}    %шрифт
\usepackage[utf8]{inputenc}    %кодировка
\usepackage[russian]{babel}    %обработка русского текста
\usepackage{graphicx}    %графика
\usepackage{minted}    %листинги кода, типа этого

\usetheme{Madrid} %тема оформления презентации, называются по городам, например, Warsaw
\usefonttheme[onlymath]{serif} %красивые формулы



\title[Короткий заголовок]{Заголовок} 
\subtitle{Подзаголовок}
\author{Автор}
\institute[Организация коротко]
{Организация полностью}
\date[Дата коротко]{Дата} % можно поставить \today -- будет текущая дата

\begin{document}

\begin{frame}  
	\titlepage  %сформировать титульный слайд
\end{frame}

\begin{frame}{Типичный заголовок слайда}
Типичный смысл слайда (его нет)
\end{frame}

\begin{frame}[fragile]{Слайд с колоночками}
Чтобы разбить слайд на колонки, нужно использовать окружение columns
\begin{columns}
\column{0.5\textwidth} Колонка, шириной 0.5 от ширины слайда. 
Этот текст является содержимым первой колонки.
\column{0.4\textwidth} Колонка, шириной 0.4 от ширины слайда
Этот текст является содержимым второй колонки.
\end{columns}
Это текст после колоночек.
\end{frame}

\begin{frame}[fragile]{Картиночки}
Для добавления картинок используется окружение figure. Но оно создаёт только поле для картинки/её названия, для добавления самой картинки нужно использовать includegraphics. \\  %  двойной бэкслэш --- насильный перенос строки
Строчка с командой includegraphics закомменчена в этом слайде, чтобы у вас точно всё скомпилировалось с первого раза. Если латех не найдёт файл картинки, то будет ошибка.
Раскомментируйте и правильно задайте имя картинки.
\begin{figure}
\centering
% здесь abacaba.png --- имя картинки. 0.7\textwidth --- ширина картинки, измеренная в единицах ширины слайда. Можно указать и в абсолютных единицах, например 10cm
%\includegraphics[width=0.7\textwidth]{abacaba.png}
\caption{Подпись к картинке}
\end{figure}
\end{frame}


\begin{frame}[fragile]{Листинги кода}
А теперь один из самых важных моментов для презентации вашего проекта --- листинги кода.
Чтобы beamer не портил форматирование кода, необходимо использовать специальное окружение --- verbatim. Оно воспроизводит содержащийся в нём текст ``буквально''
\begin{verbatim}
Текст, написанный внутри этого окружения
      воспроизводится буквально, с сохранением
            пробелов   
   и  переносов
\end{verbatim}
Однако, чтобы использовать это окружение внутри слайда, надо обязательно использовать необязательный аргумент fragile окружения frame!
\end{frame}

\begin{frame}[fragile]{Листинги кода -- upper intermediate}
Есть специальный пакет --- minted, который не только сохраняет форматирование кода, но и подсвечивает его, согласно синтаксису конкретного языка. \medskip \\ % \medskip --- команда, делающая увеличенный разрыв между строками
Код листинга на этом слайде закомменчен, потому что, если вы не настроили miktex, то код не скомпилируется. Как настроить --- в длинной версии презентации. После того, как настроите --- раскомментируйте.
%\begin{minted}{python}
%def funktsiya():
%    return "tak nazyvat' funktsii nel'zya"
%\end{minted}
Вместо python можно указать другой язык, например C++ или latex. На слайдах с листингами кода также нужно использовать fragile.
\end{frame}

\begin{frame}[fragile]{Списки}
Для нумерованных списков используется окружение enumerate. Очередной элемент списка начинается с команды item. \medskip
\begin{enumerate}
\item Текст первого пункта
\item Текст второго пункта
\item Надоело
\end{enumerate}
Для ненумерованного списка вместо enumerate напишите itemize. Списки можно вкладывать друг в друга, стиль нумерации уровней вложенности различается.
\end{frame}

\begin{frame}[fragile]{Формулки}
Вообще, \LaTeX придумал великий математик Дональд Кнут, чтобы писать в нём красивые формулки. Формулы делятся на внутристрочные и вынесенные. Первые выглядят так $S=\pi r^2$, а вторые --- так
$$
G(x-x') = \int \frac{d^4 p}{(2 \pi)^4} \frac{e^{i p_\mu (x - x')^\mu}}{p_\nu^2 - m^2 + i \epsilon}
$$
Внутристрочные  выделяются одиночными знаками доллара, а вынесенные --- двойными.
Научиться писать красивые формулы не является целью этой презентации. Однако общий смысл и правила можно почерпнуть из примеров выше. Для большего есть прекрасные руководства, в том числе на русском языке.
\end{frame}

\begin{frame}
\Huge \centering Спасибо за понимание:)
\end{frame}

\end{document}